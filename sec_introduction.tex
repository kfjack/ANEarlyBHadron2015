\section{Introduction}
\label{sec:introduction}

% KFC: let me put some blah-blah first

Measurements of $b$-hadron productions at hadronic collisions provide essential information for the
QCD interactions at high energy.  Such studies have been well carried out in several former experiments, including UA1 at CERN, as well as CDF and D0 at Fermilab. Most recent measurements have been performed by the LHC experiments, including CMS, ATLAS, and LHCb, at a center of mass energy of 7 TeV. Measurements of b-hadron production at even higher energies provided by the LHC run-II will provide a new important test of theoretical calculations. The production cross sections of b-hadrons are large hence a relatively small data set can be used to produce a sufficient measurement already, and is a critical part of the commissioning of the experiment at the new high energy run. 

This analysis is to measure the $B^+$ meson production cross sections with early LHC run-II dataset. The target channel 
$B^+ \to J/\psi K^+$ is a very well studied decay. Thus most of the properties of this decay, such as lifetime, polarization, and decay dynamics are 
all have been determined in a rather precise manner. A measurement of $B^+$ meson productions with early 13 TeV data would provide sensitive tests of the CMS system after its long shutdown period. Furthermore, the dominant background processes for this analysis consist of a 
prompt $J/psi$ combined with a random changed track can be easily suppressed using a requirement on the reconstructed decay length of the $B^+$ candidate. The CMS tracking resolution is good enough to separate the background from other $b$-hadron productions as well. Given the narrow $J/\psi$ peak and the good muon identification performance of the CMS detector, the background associated fake muon should be minimal.
The analysis can be carried out with a basic set of requirements in order to have a fast track analysis based on the minimal integrated 
luminosity.
The yields of $B$-hadron candidates are extracted using a unbinned maximum-likelihood fit to the reconstructed invariant mass distribution.
The cross sections are measured in the bins of transverse momentum and rapidity. 

In this note, we present the first measurement for the $B^+$ production cross section, through the production and decay chain $pp \to B^+X \to J/\psi K^+ X$ and $J/\psi \to \mu^+\mu^-$. A data set corresponding to an integrated luminosity of xx.xx pb$^{-1}$ in pp collisions at $\sqrt{s} = 13$~TeV. This document is arranged in the following structure: the data and simulated samples are first discussed, followed by the studies of reconstruction and selections. How to extract the signal, together with the calculations of cross sections, and their associated systematic uncertainties are introduced afterwards. Finally we conclude this analysis in the summary section. 