\section{Event Reconstruction and Selection}
\label{sec:reconstruction}

% KFC: describe the reconstruction procedure and selection here
% KFC: put T&P here as well (?)

The CMS high level trigger (HLT) evolved with increasing luminosity. As a result, some triggers may become `prescaled' to save the storage space and to prevent from exceeding the limitation of bandwidth. Within this analysis, we only use unprescaled triggers. 
This analysis highly depends on the reconstruction of $J/\psi\to{}\mu{}^{+}\mu{}^{-}$, and the `HLT\_xxxxxx' triggers are chosen. The criteria of the triggers are listed below:
\begin{itemize}
\item xxx % KFC: put the actual criteria here
\item yyy
\end{itemize}
Table~\ref{tab:selections:triggers} summarizes the information of HLT paths and the corresponding run ranges in this analysis.

\begin{minipage}{\textwidth}
    \centering
    \tabcaption{\small High level trigger paths included in this analysis and the corresponding run ranges. These 
    triggers are only differ in their output stream number, and no difference in terms of trigger criteria.}
    \label{tab:selections:triggers}
    \begin{tabular}{|c|c|c|}
        \hline
            {Trigger Path}&{[Run range]} & Luminosity \\
        \hline
            \texttt{HLT\_xxxxxx\_v1}&\texttt{[xxxxx,xxxxx]}& xx.xx $\pbinv$\\
            \texttt{HLT\_xxxxxx\_v2}&\texttt{[xxxxx,xxxxx]}& xx.xx $\pbinv$\\            
        \hline
    \end{tabular}
\end{minipage}
\FloatBarrier

After passing the HLT requirement, each $B^+$ candidate requires two muon candidates of opposite charges and a charged track (assigned as a charged kaon) to pass the baseline selection criteria discussed below.

% KFC: need to verity this...what's the new soft muon cuts now?
The official `soft muon' criteria are used in the analysis.
The muon candidates should at least meet the quality `TMOneStationTight'. The muon track should associate with
at least 6 strip hits and at least 2 pixel hits.
The inner track fitting quality is controlled by a requirement on the 
normalized $\chi{}^{2}$, which should be less than 1.8. 
The impact parameters $|dxy|$ and $|dz|$ of the muon tracks are required to be smaller than 3 cm and 30 cm, respectively.

For each pair of muon with opposite charges, a $J/\psi$ candidate is formed. 
The transverse momentum of the $J/\psi$ candidate must be greater than x.x \GeVc. % KFC: need to be tighter then the trigger cut
A vertex fitting procedure, based on the standard package KinematicVertexFitter, has been applied to the $J/\psi$ candidate using the daughter muons. The quality of their common vertex is controlled by requiring 
the vertex probability to be at least xx\%. These criteria are summarized as following:
\begin{itemize}
\item xxx % KFC: put the actual criteria here
\item yyy
\end{itemize}

Figure~xx shows the reconstructed $\mu^+\mu^-$ invariant mass distribution after the baseline selections.
The dimuon candidates which are within the $J/\psi$ mass window are included in the upcoming reconstruction of $\mu^+\mu^-K$ candidate. 
% KFC: consider a |y| dependent J/psi mass window?

Charged tracks are assumed to be kaons in the reconstruction of $B^+ \to J/\psi K^+$.
The baseline selections are applied:
candidate kaons are required to have at least 10 strip hits and at least 2 pixel hits,
and the fitting normalized $\chi{}^{2}$ should be smaller than 5. 
The candidate kaons are required to have a minimal $p_{T}$ of x.x GeV.

The reconstruction of $B^+ \to J/\psi K^+$ is carried out for each combination of $\mu^+\mu^- K^\pm$.
A kinematical constrained fit has been introduced to each combination with the KinematicConstraindVertexFitter package. 
The three tracks (two muons and one kaon) are required to be originated from a single common vertex.
A $J/\psi$ mass constraint has been introduced to improve the resolution. 
The $\chi^2$-probability of the constrained fit should be at least above xx\% for rejecting bad combinations. 
The baseline selections for $\mu^+\mu^- K^\pm$ candidates are listed below:
\begin{itemize}
\item xxx % KFC: put the actual criteria here
\item yyy
\end{itemize}

The reconstructed invariant mass distribution of $\mu\muK$ candidates after the baseline selections is shown in Fig.~xx.

